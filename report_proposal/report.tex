\documentclass{article}

\usepackage[dvipsnames]{xcolor}
\usepackage[left=2.18cm, top=1.54cm, right=2.18cm, bottom=2.54cm]{geometry}
\usepackage[UTF8, fontset=macnew]{ctex}
\usepackage{amsmath}
\usepackage{graphicx}

\pagestyle{myheadings}
\markright{第8組}

\begin{document}

Project name: 求職履歷探勘。

Member:李昀潔,呂孟芸,張譯心,陳韋霖,詹雅安。

Description: 
根據網路收集而來之求職履歷資料,透過文字探勘技術,檢驗履歷內容中,是否有有趣的特性可供我們未來求職撰寫英文履歷時參考,
因此使用文字探勘技術來探討兩個問題,
其一為驗證各求職對應之該些履歷是否具文字僵固性,其二則為找出哪些履歷文字中具有跨領域,意即對於求職,可一稿多投。

Sulotion:
第一會使用分群技術 (Clustering)來驗證各職缺對應之該些履歷是否具有內聚,
第二會計算各群文字間的距離,找出哪些資料點是在群集的邊界且跟其他一個或一個以上的群集相近。

\end{document}
